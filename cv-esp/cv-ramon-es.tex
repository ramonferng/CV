%%%%%%%%%%%%%%%%%%%%%%%%%%%%%%%%%%%%%%%%%
% Twenty Seconds Resume/CV
% LaTeX Template
% Version 1.1 (8/1/17)
%
% This template has been downloaded from:
% http://www.LaTeXTemplates.com
%
% Original author:
% Carmine Spagnuolo (cspagnuolo@unisa.it) with major modifications by
% Vel (vel@LaTeXTemplates.com)
%
% License:
% The MIT License (see included LICENSE file)
%
%%%%%%%%%%%%%%%%%%%%%%%%%%%%%%%%%%%%%%%%%

%----------------------------------------------------------------------------------------
%	PACKAGES AND OTHER DOCUMENT CONFIGURATIONS
%----------------------------------------------------------------------------------------

\documentclass[letterpaper]{twentysecondcv-esp} % a4paper for A4
\usepackage[spanish]{babel}
\selectlanguage{spanish}
\usepackage[utf8]{inputenc}
%----------------------------------------------------------------------------------------
%	 PERSONAL INFORMATION
%----------------------------------------------------------------------------------------

% If you don't need one or more of the below, just remove the content leaving the command, e.g. \cvnumberphone{}

\profilepic{fotos/foto1.jpg} % Profile picture

\cvname{Ramón} % Your name
\cvjobtitle{Ingeniero Técnico en \\ Tecnologías de \\Telecomunicación} % Job title/career

\cvdate{3 Abril 1994} % Date of birth
\cvaddress{España, Granada, Calle Carretera de la Sierra 18008} % Short address/location, use \newline if more than 1 line is required
\cvnumberphone{+34 684095402} % Phone number
\cvsite{@ramonferng} % Personal website
\cvmail{ramnfernandez@gmail.com} % Email address

%----------------------------------------------------------------------------------------

\begin{document}

%----------------------------------------------------------------------------------------
%	 ABOUT ME
%----------------------------------------------------------------------------------------

\aboutme{Soy estudiante del Máster Profesionalizante de Ingeniería de Tecnologías de la Telecomunicación en la UGR, actualmente me encuentro cursando el primer año y estoy en proceso de elaboración del Trabajo Fin de Máster sobre Ciberseguridad.
Mis intereses son bastante amplios, mi especialidad es la Telemática, aunque mi Proyecto de Fin de Grado fue de Electrónica utilizando un microchip low-cost para una aplicación IoT. También tengo interés en el diseño de software, manejando generalmente Java y C, aunque estoy iniciandome con Python. Soy, además, usuario habitual de Linux.
} % To have no About Me section, just remove all the text and leave \aboutme{}

%----------------------------------------------------------------------------------------
%	 SKILLS
%----------------------------------------------------------------------------------------

% Skill bar section, each skill must have a value between 0 an 6 (float)
\skills{
{VHDL/2},
{Arduino/2.5},
{Python/1},
{C/3},
{Java/3.5}}

%------------------------------------------------


%----------------------------------------------------------------------------------------

\makeprofile % Print the sidebar


%----------------------------------------------------------------------------------------
%	 EDUCATION
%----------------------------------------------------------------------------------------

\section{Educación}

\begin{twenty} % Environment for a list with descriptions
	\twentyitem{desde 2017}{Máster Profesionalizante de Ingeniería en Tecnologías de Telecomunicación}{\\Universidad de Granada}{}
	\twentyitem{2014 - 2017}{Grado de Ingeniería en Tecnologías de Telecomunicación, con especialidad en Telemática}{\\Universidad de Granada}{}
	%\twentyitem{<dates>}{<title>}{<location>}{<description>}
\end{twenty}


%----------------------------------------------------------------------------------------
%	 EXPERIENCE
%----------------------------------------------------------------------------------------

\section{Experiencia}

\begin{twenty} % Environment for a list with descriptions
	\twentyitem{2017}{`UGR LAN PARTY'}{\\Proyecto que consiste en la realización de diseño, despliegue e implementación de una red telemática, dirigido por el Prof. Miguel Ángel López Gordo.}{}
	%\twentyitem{<dates>}{<title>}{<location>}{<description>}
\end{twenty}


\section{Idiomas}

\begin{twenty} % Environment for a list with descriptions
	\twentyitem{2017}{Nivel B2 de Inglés}{}{Cambridge First Certificate}
    \twentyitem{2012 y 2017}{Curso de inmersión lingüística de inglés en España}{\\Beca concedida por el Ministerio de Educación, Cultura y Deportes del Gobierno de España}{Duración: +100 horas}
	\twentyitem{2011}{Curso de inmersión lingüística de inglés en Brighton y Londres}{\\Beca concedida por el Ministerio de Educación, Cultura y Deportes del Gobierno de España}{Duración: 1 mes}
\end{twenty}

\section{Habilidades}
\subsection{Software}
\begin{enumerate}
	\item \textbf{Software matemático :} MatLab, Mathematica, Maxima, Octave y Statgraphics.
	\item \textbf{Programas orientados a la Electrónica:} OrCad (Pspice), Multisim, PIC C Compiler, ISIS Professional y Quartus
	\item \textbf{Programas ofimáticos:} Microsoft Word, Access, PowerPoint y Excel.
\end{enumerate}

\subsection{Lenguajes de programación}
\begin{enumerate}
	\item \textbf{C:} Primer lenguaje usado en la carrera.
    \item \textbf{Java:} Segundo lenguaje aprendido más a fondo.
	\item \textbf{Python:} Lenguaje en proceso de aprendizaje.
	\item \textbf{VHDL:} Usado para programar FPGAs de Altera en los laboratorios de la ETSIIT e inicializándome con placas de Xilinx mediante proyectos con Vivado.
	\item \textbf{Arduino: } Usado durante la realización del proyecto IoT del TFG.
	
\end{enumerate}

\subsection{Lenguajes de marcado}
\begin{enumerate}
	\item \textbf{\LaTeX:} Usado para la entrega de trabajos en la Universidad, TFG y el desarrollo del TFM.
\end{enumerate}

\subsection{Sistemas Operativos}
\begin{enumerate}
	\item Windows: Usuario habitual
	\item Linux: Uso habitual y en continuo aprendizaje
\end{enumerate}


\end{document}
